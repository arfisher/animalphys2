% Options for packages loaded elsewhere
\PassOptionsToPackage{unicode}{hyperref}
\PassOptionsToPackage{hyphens}{url}
\PassOptionsToPackage{dvipsnames,svgnames,x11names}{xcolor}
%
\documentclass[
  letterpaper,
  DIV=11,
  numbers=noendperiod]{scrartcl}

\usepackage{amsmath,amssymb}
\usepackage{iftex}
\ifPDFTeX
  \usepackage[T1]{fontenc}
  \usepackage[utf8]{inputenc}
  \usepackage{textcomp} % provide euro and other symbols
\else % if luatex or xetex
  \usepackage{unicode-math}
  \defaultfontfeatures{Scale=MatchLowercase}
  \defaultfontfeatures[\rmfamily]{Ligatures=TeX,Scale=1}
\fi
\usepackage{lmodern}
\ifPDFTeX\else  
    % xetex/luatex font selection
\fi
% Use upquote if available, for straight quotes in verbatim environments
\IfFileExists{upquote.sty}{\usepackage{upquote}}{}
\IfFileExists{microtype.sty}{% use microtype if available
  \usepackage[]{microtype}
  \UseMicrotypeSet[protrusion]{basicmath} % disable protrusion for tt fonts
}{}
\makeatletter
\@ifundefined{KOMAClassName}{% if non-KOMA class
  \IfFileExists{parskip.sty}{%
    \usepackage{parskip}
  }{% else
    \setlength{\parindent}{0pt}
    \setlength{\parskip}{6pt plus 2pt minus 1pt}}
}{% if KOMA class
  \KOMAoptions{parskip=half}}
\makeatother
\usepackage{xcolor}
\setlength{\emergencystretch}{3em} % prevent overfull lines
\setcounter{secnumdepth}{-\maxdimen} % remove section numbering
% Make \paragraph and \subparagraph free-standing
\ifx\paragraph\undefined\else
  \let\oldparagraph\paragraph
  \renewcommand{\paragraph}[1]{\oldparagraph{#1}\mbox{}}
\fi
\ifx\subparagraph\undefined\else
  \let\oldsubparagraph\subparagraph
  \renewcommand{\subparagraph}[1]{\oldsubparagraph{#1}\mbox{}}
\fi


\providecommand{\tightlist}{%
  \setlength{\itemsep}{0pt}\setlength{\parskip}{0pt}}\usepackage{longtable,booktabs,array}
\usepackage{calc} % for calculating minipage widths
% Correct order of tables after \paragraph or \subparagraph
\usepackage{etoolbox}
\makeatletter
\patchcmd\longtable{\par}{\if@noskipsec\mbox{}\fi\par}{}{}
\makeatother
% Allow footnotes in longtable head/foot
\IfFileExists{footnotehyper.sty}{\usepackage{footnotehyper}}{\usepackage{footnote}}
\makesavenoteenv{longtable}
\usepackage{graphicx}
\makeatletter
\def\maxwidth{\ifdim\Gin@nat@width>\linewidth\linewidth\else\Gin@nat@width\fi}
\def\maxheight{\ifdim\Gin@nat@height>\textheight\textheight\else\Gin@nat@height\fi}
\makeatother
% Scale images if necessary, so that they will not overflow the page
% margins by default, and it is still possible to overwrite the defaults
% using explicit options in \includegraphics[width, height, ...]{}
\setkeys{Gin}{width=\maxwidth,height=\maxheight,keepaspectratio}
% Set default figure placement to htbp
\makeatletter
\def\fps@figure{htbp}
\makeatother

\KOMAoption{captions}{tableheading}
\makeatletter
\@ifpackageloaded{tcolorbox}{}{\usepackage[skins,breakable]{tcolorbox}}
\@ifpackageloaded{fontawesome5}{}{\usepackage{fontawesome5}}
\definecolor{quarto-callout-color}{HTML}{909090}
\definecolor{quarto-callout-note-color}{HTML}{0758E5}
\definecolor{quarto-callout-important-color}{HTML}{CC1914}
\definecolor{quarto-callout-warning-color}{HTML}{EB9113}
\definecolor{quarto-callout-tip-color}{HTML}{00A047}
\definecolor{quarto-callout-caution-color}{HTML}{FC5300}
\definecolor{quarto-callout-color-frame}{HTML}{acacac}
\definecolor{quarto-callout-note-color-frame}{HTML}{4582ec}
\definecolor{quarto-callout-important-color-frame}{HTML}{d9534f}
\definecolor{quarto-callout-warning-color-frame}{HTML}{f0ad4e}
\definecolor{quarto-callout-tip-color-frame}{HTML}{02b875}
\definecolor{quarto-callout-caution-color-frame}{HTML}{fd7e14}
\makeatother
\makeatletter
\makeatother
\makeatletter
\makeatother
\makeatletter
\@ifpackageloaded{caption}{}{\usepackage{caption}}
\AtBeginDocument{%
\ifdefined\contentsname
  \renewcommand*\contentsname{Table of contents}
\else
  \newcommand\contentsname{Table of contents}
\fi
\ifdefined\listfigurename
  \renewcommand*\listfigurename{List of Figures}
\else
  \newcommand\listfigurename{List of Figures}
\fi
\ifdefined\listtablename
  \renewcommand*\listtablename{List of Tables}
\else
  \newcommand\listtablename{List of Tables}
\fi
\ifdefined\figurename
  \renewcommand*\figurename{Figure}
\else
  \newcommand\figurename{Figure}
\fi
\ifdefined\tablename
  \renewcommand*\tablename{Table}
\else
  \newcommand\tablename{Table}
\fi
}
\@ifpackageloaded{float}{}{\usepackage{float}}
\floatstyle{ruled}
\@ifundefined{c@chapter}{\newfloat{codelisting}{h}{lop}}{\newfloat{codelisting}{h}{lop}[chapter]}
\floatname{codelisting}{Listing}
\newcommand*\listoflistings{\listof{codelisting}{List of Listings}}
\makeatother
\makeatletter
\@ifpackageloaded{caption}{}{\usepackage{caption}}
\@ifpackageloaded{subcaption}{}{\usepackage{subcaption}}
\makeatother
\makeatletter
\@ifpackageloaded{tcolorbox}{}{\usepackage[skins,breakable]{tcolorbox}}
\makeatother
\makeatletter
\@ifundefined{shadecolor}{\definecolor{shadecolor}{rgb}{.97, .97, .97}}
\makeatother
\makeatletter
\makeatother
\makeatletter
\makeatother
\ifLuaTeX
  \usepackage{selnolig}  % disable illegal ligatures
\fi
\IfFileExists{bookmark.sty}{\usepackage{bookmark}}{\usepackage{hyperref}}
\IfFileExists{xurl.sty}{\usepackage{xurl}}{} % add URL line breaks if available
\urlstyle{same} % disable monospaced font for URLs
\hypersetup{
  pdftitle={Lab 3 Worksheet: Peripheral Circulation and The Dive Response},
  pdfauthor={Marguerite Butler},
  colorlinks=true,
  linkcolor={blue},
  filecolor={Maroon},
  citecolor={Blue},
  urlcolor={Blue},
  pdfcreator={LaTeX via pandoc}}

\title{Lab 3 Worksheet: Peripheral Circulation and The Dive Response}
\author{Marguerite Butler}
\date{2023-09-01}

\begin{document}
\maketitle
\ifdefined\Shaded\renewenvironment{Shaded}{\begin{tcolorbox}[sharp corners, breakable, borderline west={3pt}{0pt}{shadecolor}, frame hidden, boxrule=0pt, enhanced, interior hidden]}{\end{tcolorbox}}\fi

\renewcommand*\contentsname{Table of contents}
{
\hypersetup{linkcolor=}
\setcounter{tocdepth}{3}
\tableofcontents
}
\begin{tcolorbox}[enhanced jigsaw, toptitle=1mm, colframe=quarto-callout-tip-color-frame, rightrule=.15mm, breakable, left=2mm, colbacktitle=quarto-callout-tip-color!10!white, arc=.35mm, bottomrule=.15mm, coltitle=black, bottomtitle=1mm, titlerule=0mm, opacitybacktitle=0.6, opacityback=0, toprule=.15mm, colback=white, title=\textcolor{quarto-callout-tip-color}{\faLightbulb}\hspace{0.5em}{Notes}, leftrule=.75mm]

\begin{itemize}
\tightlist
\item
  Please pay attention to the \textbf{intellectual content} contained in
  each section of this \textbf{worksheet}.
\item
  Our goal here is the practice logically laying out our \textbf{ideas}
  focusing on \textbf{\emph{physiological mechanisms}}.
\item
  Sections (1) and (2) together are designed to guide you to building
  strong hypotheses, and (3) and (4) are mini-results and
  mini-discussion sections, respectively.
\end{itemize}

\end{tcolorbox}

\hypertarget{statements-of-purpose}{%
\subsection{\texorpdfstring{\textbf{(1) Statements of
Purpose}}{(1) Statements of Purpose}}\label{statements-of-purpose}}

\textbf{Use these statements of purpose to frame your thinking about
this lab, complete them as needed:}

\emph{Distance from heart} -- ``We demonstrate the predicted drop in
blood pressure with distance from the central pump. We explore whether
there is a greater drop in systolic or diastolic pressure with
distance'' (in a full lab report you would set up your ideas for why it
might go either way).

\emph{Effect of gravity} -- ``We demonstrate the action of gravity on
blood pressure. As vertebrates possess a closed circulatory system,
vertical height should contribute to blood pressure in predictable
ways.''

\emph{Dive response} -- ``When diving, peripheral circulation should be
reorganized to reduce cardiac output and increase peripheral
vasoconstriction, called the dive response. We aim to observe the dive
response by measuring
\_\_\_\_\_\_\_\_\_\_\_\_\_\_\_\_\_\_\_\_\_\_\_\_.''

\emph{Stimulus for dive response} -- ``We will test potential triggers
for the dive response including
\_\_\_\_\_\_\_\_\_\_\_\_\_\_\_\_\_\_\_\_\_\_\_\_\_\_\_\_.''''

\hypertarget{expectations-demonstrate-your-mastery-of-the-mechanisms}{%
\subsection{\texorpdfstring{\textbf{(2) Expectations demonstrate your
mastery of the
mechanisms}}{(2) Expectations demonstrate your mastery of the mechanisms}}\label{expectations-demonstrate-your-mastery-of-the-mechanisms}}

\begin{tcolorbox}[enhanced jigsaw, toptitle=1mm, colframe=quarto-callout-tip-color-frame, rightrule=.15mm, breakable, left=2mm, colbacktitle=quarto-callout-tip-color!10!white, arc=.35mm, bottomrule=.15mm, coltitle=black, bottomtitle=1mm, titlerule=0mm, opacitybacktitle=0.6, opacityback=0, toprule=.15mm, colback=white, title=\textcolor{quarto-callout-tip-color}{\faLightbulb}\hspace{0.5em}{Notes}, leftrule=.75mm]

\emph{After identifying the mechanisms, communicate (at least hint at)
what you expect to see in the data if the hypothesized mechanisms are
occurring. That shows deeper understanding.}

\end{tcolorbox}

If these mechanisms are occurring in your data, what would you expect to
see? Be specific and relate back to observable parameters: (explain how
you controlled for confounding parameters, if appropriate). \emph{In a
full lab report this would be written in paragraph form, in a narrative
style.}

\begin{itemize}
\item
  Distance from heart
\item
  Effect of gravity
\item
  Dive response
\item
  Stimulus for dive response
\end{itemize}

\hypertarget{mini-results}{%
\subsection{(3) Mini Results}\label{mini-results}}

\emph{Display your results by including either \textbf{a figure or a
table} for each important result} (you may work together with your group
to produce the figures or tables). \emph{\textbf{Write one sentence}
pointing out what your data actually shows for each display item.}

\hypertarget{mini-discussion}{%
\subsection{(4) Mini Discussion:}\label{mini-discussion}}

\textbf{Wrap-up:} In paragraph form, briefly \textbf{Discuss} the
\textbf{main take-aways} that you learned from these experiments on
peripheral circulation and the dive response. Use specific results that
back up your statements or speculate on the significance of the results.
Organize by hypotheses above.

\textbf{Individual assignment.} Text must be your own, but you may work
together with your group to produce figures/tables. You may edit this
sheet. Submit by hard copy next week.



\end{document}
